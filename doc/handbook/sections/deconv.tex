
\section{Tuning the deconvolution}

\begin{lstlisting}

==============================================
FICHIER POUR PSFSIM_SIG_MINI_new4.f
==============================================
=======================================================
| Nombre d'images                        |1
|----------------------------------------|-------------
| Nombre de sources ponctuelles          |$nbrpsf$
|----------------------------------------|-------------
| Est-ce une image recompos�e (y/n)      |y
=======================================================
| FWHM finale des sources                |2. 				# this is the width of the final resoltuion you want to reach in the deconvolved image.
|----------------------------------------|-------------
| FWHM de G                              |$resolpix$ 			# this is the resolution of the input image (i.e. the seeing) : but usually we put higher values, so that it works better.
|(r�solution image de d�part en GP)      |
|----------------------------------------|-------------
| FWHM pour le terme de lissage          |7.0 				# sigma of the filter is used in the smoothing (the sigma, not the lambda multplier !) -> put 3.0
=======================================================
| Nb d'it�rations entre 2 sauvegardes    |1000
|----------------------------------------|-------------
| Nb d'it�rations max. pour minimi moffat|3000
|----------------------------------------|-------------
| Nb d'it�rations max. pour minimi fond  |1000
|----------------------------------------|-------------
| Valeur deltaCHI min entre 10 it�rations|1.E-2				# stop ciretrion : put to 1E-40 if you never want the programm to stop because of this.
=======================================================
| PARAMETRES POUR LE FOND						# this is now replaced by the famous "gaussian"
=======================================================			# so : we do not want to use this feature : to stop it, put 
| Rayon ini pour zone fond � modifier    |100				# same value for initial radius and the final radius.
| Elargissement par �-coups              |10				#
| Rayon de lissage pour la parabole      |5				#
| Rayon maximum � traiter                |100				#
=======================================================
| MINIMI MOFFAT
=======================================================
| Pas pour la moffat:         B1 =       |0.0004			# B1 and B2 are some x and y anisotropy
|                             B2 =       |0.0004
|                             B3 =       |0.00002			# B3 has to do with the rotation, not related to B1 and B2
|                             Beta =     |0.02				# this is the moffat exponent.
|----------------------------------------|-------------
| Pas pour les param�tres du  plan       |0.0				# keep very small (1.E-7) : this is a step related to the construction of a flat background. 
|----------------------------------------|-------------
| Pas pour les d�calages      en X =     |0.0				# step for movements from one ccd image to the other : but we use stars from one image
|                             en Y =     |0.0				# so put this to 0 : the PSF-stars from one image will still adapt for shifts !
|----------------------------------------|-------------
| Pas pour les sources        A (%) =    |50.				# Step on intensity : 1% is a good starting point. the exact value is the peak ADU value measured on the input image.
|                             C =        |0.1				# this is a step in pixels of movement of the Moffat center
=======================================================
| MINIMI FOND
=======================================================
| Pas sur le Fond                        |1.				# this is not in ADU !!!! it is a multiplicative factor that relates
| Constante                              |0.				# the background variation to the convergence speed. keep 1 to 4 ... put constant to 0. It's a constant on the step, not on the background. Nobody understands this, keep it to zero !!!
|------------------------------------------------------
| FWHM initiale de la gaussienne         |$fondgaussini$		# this is the evolution of the smooting, replaces the parabola. Initial width
|------------------------------------------------------
| FWHM finale de la gaussienne           |60.				# final width (good values : initial 30, final 50)
|------------------------------------------------------
| Nombre d'it�rations pour la gaussienne |1000				# number of iterations
|------------------------------------------------------
| R�sidu minimum (resi_min)              |0.3
=======================================================
| VALEURS INITIALES
=======================================================
| Valeur du lambda  (lambda(1) ... lambda(ni))        |			# we use only one image as source for our PSFs, so we have only 1 lambda
|------------------------------------------------------
 1.									# 
|------------------------------------------------------
| Valeur initiale du fond                             |
|------------------------------------------------------
 0.									# this is zero as we subtracted the background : keep it 
|------------------------------------------------------
| Minimisation de la moffat                           |
| {y ou n         }  ou  {i        }                  |
| {b1 b2 b3 beta  }  ou  {nom_image}  * ni            |
|------------------------------------------------------
 y									# do not touch this; 0.0 means that he will use the initial values.
 0.  0.  0.  0.
|------------------------------------------------------
| Param�tres initiaux du plan (alpha beta gamma) * ni |			# parameters for the flat background : 0 means that the background is horizontal.
|------------------------------------------------------
 0.  0.  0.
|------------------------------------------------------
| Param�tres initiaux des d�calages                   |
| ( deltaX deltaY ) * ni                              |
|------------------------------------------------------
 0.  0.									# keep them to zero
|------------------------------------------------------
| Param�tres initiaux des sources                     |
| a(1) ... a(ni) }                                    |
| cx cy          } * ns                               |
|------------------------------------------------------
$paramsrc$


format :
central peak in big pixel as ADU
x_center, y_center



\end{lstlisting}
